%
% latex template for SAS documentation 
% name: sas.tex 
% date last modified: 6 may 2018
% modified by: jerry
% 
%
\documentclass{article}

\usepackage{caption}
\usepackage[margin=1in]{geometry}
\usepackage{graphicx}
\usepackage{hyperref}
\usepackage{float}
\usepackage{tabularx}
\usepackage{titling}

\begin{document}

\title{
	\includegraphics{images/um_logo.png} \\
	\vspace{0.1in}
	CSC431 \\
	\vspace{0.2in}
	\textbf{Download of Public-facing Data} \\
	\large System Architecture Specification \\
	Team \#3
}

\author{
	Jerry Bonnell
	\and Gururaj Shriram
	\and Lixiong Liang
	\and Heyu Yao
	\and Re Chang
}

\date{}
\maketitle

\clearpage
\section*{Version History}

\begin{tabularx}{\textwidth}{| l | l | X | l |}
	\hline
	\textbf{Version} & \textbf{Date} & \textbf{Author(s)} & \textbf{Change Comments} \\
	\hline
	2 & \today & Jerry Bonnell, Re Chang, Lixiong Liang & Final Draft \\ 
	\hline
	1 & April 4, 2018 & Jerry Bonnell and Gururaj Shriram & First Draft \\
	\hline
\end{tabularx}

\clearpage
\tableofcontents

\clearpage
\listoffigures
\listoftables

\clearpage

\section{System Analysis}

\subsection{System Overview}

The System will be comprised of four main parts: a data manager, data packager, data converter, and REST controller. The data manager will control the connection to the database and handle database queries.\footnote{This is later to be replaced by a call to a database API.} The data packager will package all requested pieces of data into a compressed file if multiple pieces of data are requested. The data converter will convert data from the database into the user's desired format. This format can be any of the supported file formats: kml, esri Shapefile, csv, or geoJSON. The REST controller will manage all of the other aspects of the system and manage the end-to-end pipeline of the downloading process. \\

This pipeline is as follows: (1) the browser submits a download request to the download server which the REST controller listens to, (2) the controller initiates the process of querying data from the database based on the list of layers provided by the browser request, (3) the data is then converted to the requested output format, and (4) the server packages data before submitting it to the browser for local download. \\ 

The three-tier architecture will be used for this project. The client-server tier allows the download server to handle requests from the search team. The database-centric tier queries the database to obtain geospatial and multimedia data. The business layer performs the conversion and archival of this data, before the client-server tier sends the packaged data back to client.   

\subsection{System Diagram}

\begin{figure}[H]
	\begin{center}
		\caption{Download System Diagram}
		\includegraphics[width=\textwidth]{images/component.pdf}
	\end{center}
\end{figure}

\clearpage

\subsection{Actor Identification}

There are three types of human actors: unauthorized users, authorized users, and the administrators. Unauthorized users may only download public pieces of data unless they are granted permission to specific pieces of data from an administrator. Authorized users are allowed to download any piece of data. Administrators are able to download any piece of data as well as approve requests for unauthorized users to download private pieces of data. The system must also support interaction with a non-human actor, i.e., an API, in order to retrieve
required data. 

\subsection{Design Rationale}

\subsubsection{Architectural Style}

This system, which is part of a larger system, solely exists on the back-end. In designing this, we are using the 3-tier architectural style because our system is a back-end API. This lends itself to a natural division into three tiers: client-server, business, and database-centric. In the client-server tier, the REST controller receives user input, validates the input, and passes that on to be processed by the business tier before responding to the client's request. The business tier requests data from the database-centric tier, converts that data and, finally, archives the converted data and sends it back to the client-server tier. The sole purpose of the database-centric tier is to interface with the database. 

\subsubsection{Design Pattern(s)}

Express is a flexible Node.js framework that provides a robust set of features. We predict that the following design patterns will be primarily used in the following ways: \\

\begin{itemize} 
	\item Factory: To abstract objects that may share common tasks, such as data management and parsing objects
	\item Singleton: To maintain a single, consistent reference, such as a database reference
	\item Middleware: To handle flows that have middleware functions in the system's request-response cycle, such as in the data conversion process
	\item Stream: To process large amounts of flowing data, such as in the file writing/core download process
\end{itemize}


\subsubsection{Framework}

The web application will run on the \texttt{Node.js} JavaScript engine on the \texttt{Express} framework. \texttt{Express} allows for the successful implementation of a REST API on the backend and will facilitate a lightweight, efficient download pipeline. \\

\textbf{Links:}
\begin{itemize} 
	\item \url{https://nodejs.org}
	\item \url{https://expressjs.com}
\end{itemize}

\clearpage

\subsection{Cross-cutting Requirements}

We have identified two cross-cutting requirements that are involved in the download pipeline:  

\begin{enumerate}
\item Performing a \textbf{download}.

Performing a download requires the implementation of a download button on the UI, which calls our download web server, and, at some point, makes contact with the storage database. Therefore, this is a cross-cutting requirement involving three teams that are heavily dependent on each other: search/UI, download, and database. \\

\item Gathering \textbf{geospatial data} from a database.

Further, from the list of layers, our server must be able to gather geospatial data from the Postgres database. This is cross-cutting between the database and our download team.

\end{enumerate}

\clearpage

\section{Functional Design}

Here we present the sequence diagram for the download use case. This
outlines the core functionality for this system. Following it are 
clarifications regarding the diagram. 

\begin{figure}[H]
	\begin{center}
		\caption{Sequence Diagram for the Download Use Case}
		\includegraphics[width=\textwidth]{images/sequence_diagram.pdf}
	\end{center}
\end{figure}

\begin{itemize}
	\item When a user clicks the download button on the search page, an HTTP 
	POST request is sent to our web server, which calls on our REST controller 
	to handle the request and retrieve the required data. 
	\item The controller will then call FileWriter to generate a unique 
	name for the request directory and the files contained inside. Using this 
	name, a folder is created. 
	\item For each layer, a call is made to DataManager to generate queries 
	that fetch the desired data (both multimedia and geospatial) for the given 
	references from the storage database. On the production machine, a call 
	would instead be made to the backend's API for retrieval. 
	\item The database returns data in its stored format, and 
	FileWriter writes this data to file.
	\item For geospatial data, DataConverter is invoked to convert the data  
	from its stored format to the desired output format (if necessary). No 
	conversion is necessary for returned multimedia data.
	\item GeoJSON files are deleted from the directory if not requested. 
	At the very end, DataPackager is called to package all of the 
	download files into a single URI that is returned to the search.

\end{itemize}

\clearpage

\section{Structural Design}

The following class diagram represents an overview of the component breakdown in our system. 
A primary focus for this design is modularity.  

\begin{figure}[H]
	\begin{center}
		\caption{Download Class Diagram}
		\includegraphics[width=\textwidth]{images/class_diagram.pdf}
	\end{center}
\end{figure}

An asterisk is here required. The DataManager outlined here  
interfaces with the database and, as such, carries the responsibility for 
generating queries. Strictly speaking, this does not adhere to the requirements. 
In its place, a call should be made to the backend's API for retrieval
of the data. We expect most of the DataManager's methods to be deprecated 
and replaced by an API call on the production machine.

\end{document}
	
